\documentclass[a4paper,12pt]{article}
\usepackage[utf8]{inputenc}
\usepackage[brazil]{babel}

\usepackage{setspace}
\setstretch{1.5}
\usepackage[lmargin=3cm,tmargin=3cm,rmargin=2cm,bmargin=2cm]{geometry} %Formato que lembra a ABNT
\usepackage[T1]{fontenc} %Ajusta o texto que vem de outras fontes
\usepackage{graphicx,xcolor,enumerate}


\begin{document}








\section{FUNDAMENTAÇÃO TEÓRICA}
\subsection{TECNOLOGIA E EMPRENDEDORISMO}
\subsection{DIAGRAMA DE CLASSES}
Diagrama de classes é um tipo de diagrama UML muito usado para documentar arquiteturas de sistemas. Ao criar um diagrama de classes em um projeto de software orientado a
objetos, o mesmo irá conter as classes que normalmente serão convertidas em classes e objetos reais quando o código é gravado
Destaca-se a sua composição padrão, a qual é dividida em três partes: a parte superior,
que contém o nome da classe, a parte do meio, que contém os atributos da classe, e a parte
inferior, que inclui os métodos da classe [].
Decerto existem alguns benefícios em utilizar o diagrama de classes, dentre eles, ilustrar
modelos de dados para sistemas, sejam eles simples ou complexos, e expressar visualmente as
necessidades de um sistema [].
\subsection{CASOS DE USO}
Os casos de usos são um tipo de diagrama UML [], comumente utilizado para detalhar
os usuários e suas interações com o sistema. Nota-se algumas vantagens no uso do diagrama,
como o auxílio na representação de metas de interações entre sistema e usuário, na definição e
organização dos requisitos funcionais do software, especificação do contexto, e modelagem do
fluxo básico dos eventos.
Existe ainda uma estrutura de especificação do caso de uso, com mais detalhes sobre
o mesmo. É possível modificar a estrutura de tópicos conforme necessário em cada situação.
Alguns dos tópicos são []: nome do caso de uso, descrição, pré-condição, fluxos principal e
alternativo.
\subsection{FIGMA}
O Figma é um editor feito para prototipagem de projetos de design, com a existência de
planos gratuitos e pagos. A ferramenta foi disponibilizada pela primeira vez ao público em 2016
    [] e, desde então, tanto profissionais de design como amadores podem utilizar o Figma.
Ademais, existem muitas funcionalidades na ferramenta [], as que se destacam são:
versionamento automático, painel de camadas e objetos, biblioteca compartilhável de componen-
tes, possibilidade de compartilhar o projeto e colocá-lo em modo de apresentação, e também
permite criar fluxo de navegação entre telas.
\subsection{VISUAL STUDIO CODE}
O Visual Studio Code é um editor de código-fonte lançado em 2015 e desenvolvido pela
empresa Microsoft []. A IDLE é baseada no framework Electron e está disponível para os
sistemas operacionais Windows, Linux e macOS.
É uma ferramenta que vem com suporte integrado para algumas linguagens de programa-
ção, bem como uma variada lista de extensões para outras linguagens []. Por outro lado, tem
um ambiente de depuração, e ainda conta com um controle de versionamento com o Git.
\subsection{GIT E GITHUB}
As duas ferramentas, normalmente, são utilizadas juntas. Primeiramente, o Git é um
sistema de controle de versão distribuído [], e um recurso que o destaca dentre os outros do
mesmo tipo é seu recurso de ramificações, as quais podem ser independentes entre si, ou seja,
pode ser feita uma alteração em uma ramificação específica e não vai haver mudanças nas outras.
Já o GitHub é um local de hospedagem de código-fonte e arquivos com controle de
versão utilizando o Git []. É uma ferramenta muito popular, com a hospedagem de mais de
100 milhões de repositórios em seu sistema []. Esse renome acontece por todos os usuários
poderem acompanhar e gerenciar mudanças nos seus projetos em tempo real []. E, assim como
nas redes sociais, os usuários podem seguir outros, para acompanhar os projetos e até mesmo
participar dos mesmos.
\subsection{Dart}
Dart é uma linguagem de programação usada para desenvolver aplicações de várias plataformas a partir do mesmo código fonte. Ela foi criada pela empresa Google em 2011 \cite{dart}.
É uma linguagem fortemente tipada que auxilia no desenvolvimento de software e evitar erros de sintaxe.Porém, também possui desvantagens como ter pouco tempo de mercado
\subsection{FLUTTER}
Flutter é um framework de código aberto desenvolvido pela Google em 2015. É baseado
na linguagem de programação Dart, o que possibilita criação de aplicações para mobile, web e
desktop \cite{flutter}.
Uma grande vantagem do Flutter é sua alta produtividade, já que a mesma base de código
vai servir tanto para um aplicativo Android como para um iOS \cite{geekhunter}. Assim, não há necessidade
de se preocupar com arquitetura e configurações, e sim somente com a aplicação em si. Por isso,
grandes empresas utilizam o framework como o Nubank, iFood, Ebay e o próprio Google \cite{geekhunter}.
\subsection{JAVASCRIPT}
JavaScript (frequentemente abreviado como JS) é uma linguagem de programação
interpretada estruturada, de script em alto nível com tipagem dinâmica fraca e
multiparadigma (protótipos, orientado a objeto,
imperativo e funcional) \cite{javascript}.
\subsection{TYPESCRIPT}
TypeScript é uma linguagem de programação de código aberto desenvolvida pela
Microsoft. É um superconjunto sintático estrito de JavaScript e adiciona tipagem
estática opcional à linguagem. Tipos fornecem uma maneira de descrever a forma de
um objeto, fornecendo melhor documentação e permitindo que o TypeScript valide
se seu código está funcionando corretamente.
Como TypeScript é um superconjunto de JavaScript, os programas JavaScript existentes também são programas TypeScript válidos \cite{typescript}.
\subsection{NODE}
Node.js é um ambiente de execução JavaScript que permite executar aplicações desenvolvidas com a linguagem de
forma autônoma, sem depender de um navegador. Com ele, é possível criar
praticamente qualquer tipo de aplicações web, desde servidores para
sites estáticos e dinâmicos, até APIs e sistemas baseados em microserviços \cite{node}.
\subsection{API}
A sigla API deriva da expressão inglesa Application Programming Interface que, traduzida para o português, pode ser compreendida como uma interface de programação de aplicação. Ou seja, API é um conjunto de normas que possibilita a comunicação entre plataformas por meio de uma série de padrões e protocolos.
Por meio de APIs, desenvolvedores podem criar novos softwares e aplicativos capazes de se comunicar com outras plataformas. Por exemplo: caso um desenvolvedor queira criar um aplicativo de fotos para Android, ele poderá ter acesso à câmera do celular através da API do sistema operacional, sem ter a necessidade de criar uma nova interface de câmera do zero \cite{api}.
\subsection{REST}
Na informática e engenharia de software, Representational State Transfer (abreviado REST), em português Transferência de Estado Representacional, é um estilo de arquitetura de software, criado em 2000 por Roy Fielding, que define um conjunto de restrições a serem usadas para a criação de um tipo especial de serviços-Web, denominados Web services RESTful, que fornecem interoperabilidade entre sistemas de computadores na Internet; RESTful permite que os sistemas solicitantes acessem e manipulem representações textuais de recursos da Web usando um conjunto uniforme e predefinido de operações sem estado (requisição e resposta independentes). Outros tipos de Web services, como Web services SOAP, expõem seus próprios conjuntos de operações arbitrários\cite{rest}.
\subsection{POSTGRES}


























\end{document}
