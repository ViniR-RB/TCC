\documentclass[a4paper,12pt]{article}
\usepackage[utf8]{inputenc}
\usepackage[brazil]{babel}
\usepackage{csquotes}
\usepackage{multirow}
\usepackage{booktabs}
\usepackage{setspace}
\usepackage{url}
\usepackage{caption}
\usepackage{biblatex}
\addbibresource{referencias.bib}
\usepackage[colorlinks=true,linkcolor=black,urlcolor=blue,citecolor=blue]{hyperref}
\setstretch{1.5}
\usepackage[lmargin=3cm,tmargin=3cm,rmargin=2cm,bmargin=2cm]{geometry}
\usepackage[T1]{fontenc}
\usepackage{graphicx,xcolor,enumerate}


\begin{document}
% Lista de Figuras
\listoffigures
\clearpage
% Lista de Tabelas
\listoftables
A lista de tabelas pode ser encontrada na página~ \ref{tab:casos-de-uso}.
\clearpage
% INTRODUÇÃO
\section{INTRODUÇÃO}
\subsection{CONTEXTUALIZAÇÃO}
O Brasil tem atualmente mais de um smartphone por habitante, segundo levantamento anual divulgado pela FGV. São 242 milhões de celulares inteligentes em uso no país, que tem pouco mais de 214 milhões de habitantes, de acordo com o IBGE \cite{celulares}.
Desse Modo é mais fácil para empresas criarem seus próprios aplicativo de pagamento de serviços, pois entrega aos seus clientes uma maior integração e facilidade no momento do pagamento,dos serviços prestados.
Além de facilitar a vida dos usuários e também das empresas, fazendo com que o foco de uma empresa seja em outras áreas que possibilitam um maior conforto na hora de gerência/pagamento dos serviços prestado
\subsection{JUSTIFICATIVA}
Na Contemporaneidade  é muito difícil para pessoas que tão iniciando na vida do empreendedorismo ou já são micro-empresas nascerem grande e concorrerem com outras empresas, logo para se destacar no mercado é obrigatório trabalhar no diferencial de que as  concorrentes não oferecem, mantendo assim um padrão de qualidade, logo esse software impacta positivamente na qualidade de atendimento desse empreendedor ou da micro empresas que estão procurando ter dinamicidade na empresa ou no pequeno negócio.
\subsection{OBJETIVOS}
\subsubsection{Objetivo Geral}
Solução de Software para gestão de pagamento de um serviço,com propósito de acelerar,agilizar e tornar mais interativo o pagamentos de um serviço.
\subsubsection{Objetivos Específicos}

\begin{itemize}
    \item Gestão de Pagamentos desacoplado (e com possiblidades de novas integrações de pagamentos) que possibilita as ações de criação, emissão e acompanhamento de compensação em diversas modalidades, tais como Cartão, PIX e Boleto.
    \item Disponibilizar um Sistema Mobile para tornar o aplicativo mais acessível para Empresas e  Usuários.
    \item Aproveitar o crescente movimento dos celulares para disponibilizar todo o poder e liberdade nos pagamento.
\end{itemize}

\section*{Resumo}
\hspace{0.5cm}Neste capítulo,foi apresentado uma introdução sobre os aumentos do uso de celulares no Brasil.Logo depois,foi mmostrada a justificativa do trabalho,que se concentra na ideia dos benéficios a união  dos temas celulares e o como a tecnologia pode trazer benéficios para empresas de menores porte concorrerem com as grandes no mercado.Por
último, foram pontuados os objetivos geral e específicos do projeto.

\hspace{0.5cm}Os próximos capítulos estão apresentados na seguinte sequência:

\begin{itemize}
    \item \textbf{Fundamentação Teórica:} Capítulo voltado para a exibição dos conceitos teóricos
          utilizados no desenvolvimento da solução proposta no trabalho;
    \item \textbf{Metodologia:} Capítulo dedicado a apresentação da metologia utilizada na criação
          do projeto da solução, através do levantamento de requisitos do software proposto,
          da listagem das ferramentas que foram e das que serão utilizadas, e apresentação do
          projeto de aplicação.
    \item \textbf{Conclusão:}Capítulo direcionado a conclusão do projeto, com a sugestão de trabalhos
          futuros, os quais se resumem a iniciar o desevolvimento prático da solução.
\end{itemize}
















\end{document}
